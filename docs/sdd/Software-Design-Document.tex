\documentclass[letterpaper]{article}
\usepackage{tempora}

\begin{document}
\title{\textbf{hashscript}: hash for use in written text}
\author{
Korey Hinton
\and
Kasey Hinton
}
\date{October 2022}

\maketitle

\newpage

\section*{Copyright restrictions}
\textbf{Copyright 2022 Korey Hinton and Kasey Hinton. \\ All Rights Reserved.}

\newpage

\tableofcontents

\newpage

\section{Abstract}
%\input{file.tex}
The word hashscript can refer to the textual notation
that represents textual transform to embed numeric hash digits,
and the word hashscript can also refer to the name
of the JavaScript library containing such a
textual transform implementation.

A series of numbers appearing in a segment of text,
if written in a way that the chosen number would not
change the meaning outside the text segment,
could store parts of a numeric version of a hash.
Each hash digit would need to be greater than or equal to two to use
with plural nouns.

\section{hashscript Notation}

The hashscript
notation represents numbers with placeholders
to use as dynamic text that gets replaced
with numeric hash digits. The dynamic form
uses a variable syntax (i.e., ``\$A'') before hashing
the text segment. After hashing, each variable
placeholder gets replaced with the corresponding hash
digit number (i.e., ``\$A -> 6''; \$B -> 9'').

An example text segment in hashscript notation
could be \textit{``The team spotted \$A dolphins, \$B orcas,
and \$C jellyfish.''}

\section{hashscript Hash Format}

The format of the hashscript hash would be space-separated
numeric hash digits, i.e., \begin{verbatim}6 10 11 9 4 3 10 10 2 8 11 5 4\end{verbatim}

\section{hashscript Implementation (JavaScript library)}


\end{document}
